\documentclass{article}
\usepackage{graphicx} % Required for inserting images
\usepackage{polyglossia}
\setmainlanguage{english}
\usepackage{hyperref}
\setotherlanguages{hindi}
\newfontfamily\devanagarifont[Script=Devanagari]{Noto Serif Devanagari}

\title{ITL Project Part-1}
\author{Ojas Kataria}

\begin{document}

\maketitle

\section{Text}
Source of text: \href{https://www.storywithanvi.com/best10-moralstories}{\begin{hindi} सोने की चिड़ियाँ \end{hindi}}\\

\begin{hindi}
एक बार ढोलकपुर जंगल में कू-कू कोयल बहुत ही मीठा गाना गा रही थी। मैं हूँ कू-कू कोयल।  गाती हूँ सुंदर मीठा-मीठा गाना। तुम सब भी सुनकर आ जाना, आ जाना आ जाना। उसकी आवाज दूर-दूर तक सबको सुनाई दे रही थी। तभी उसकी आवाज सुन कर एक जादूगर वहाँ आता है, और उससे कहता है। वाह, कू-कू कोयल तुम्हारी आवाज तो बहुत मीठी है। बिल्कुल खरा सोना है। सोना.. हाँ । तभी जादूगर आबरा-का डाबरा बोलता और कू-कू कोयल से कहता है, तुमअब से जब भी गाना गाओगी और गाते समय अगर तुम्हारे मुंह से एक भी बूँद पानी की नीचे गिर गई तो वो बूंद सोने में बदल जाएगी, लेकिन तुम्हें एक बात का हमेशा ख्याल रखना होगा कि तुम कभी ऐसे लोगों के सामने नहीं गाना जो तुमको किसी भी तरह से नुकसान पहुचाएं।

ऐसा बोलने के बाद जादूगर वहाँ से चला जाता है। इधर कू-कू कोयल फिर से गाना गाने लगती है। तभी वहाँ पर चम्पा लोमड़ी आती है और वो कू-कू कोयल का गाना सुनने लगती है। गाना गाते-गाते कू-कू कोयल के मुहँ से पानी गिरता है। पानी के नीचे गिरते ही वो सोने में बदल जाता है। चम्पा लोमड़ी अपने सामने सोना देख कर बहुत खुश हो जाती है और झट से सोना उठाती है फिर वहाँ से बहुत तेज दौड़ लगाते हुए अपने घर पहुँचती है। इधर कू-कू कोयल गाना गाती रहती है। तभी वहाँ पर एक शिकारी आता है। शिकारी पेड़ के नीचे बैठकर कू-कू कोयल का गाना सुनने लगता है। कू-कू कोयल गाना गा ही रही थी कि फिर से उसके मुहँ से पानी नीचे गिरता है। मुहँ का पानी नीचे पहुंचते ही सोने में बदल जाता है। अपने सामने शिकारी सोना देख कर बहुत खुश हो जाता है। वो झट से उसे उठा लेता है। तभी शिकारी फिर देखता है कि ऊपर से पानी की एक ओर बूंद गिरी जो सोने में बदल गई। शिकारी फिर खुश होता है और उसे भी अपने पास रख लेता है। ऐसे करते हुए कई बूंदे ऊपर से नीचे गिरती है जो सोने में बदलती जाती है। शिकारी मन में सोचता कि शायद आसमान से सोने की बारिश हो रही है और ये सोचते हुए वो एक-एक कर के सोना उठाता जाता है। तभी शिकारी कि नजर गाना गा रही कू-कू कोयल पर जाती है। और वो देखता है कि ये सोने वाला पानी तो कू-कू कोयल के मुहँ से निकल रहा। फिर वो बिना देरी किये कोयल को पकड़ने के बारे में सोचता है क्योंकि कू-कू कोयल को पकड़ कर वो बन जाएगा अब बहुत-बहुत अमीर। इधर कू-कू कोयल गाना गाने में इतना खो जाती है कि उसे इस बात का भी ध्यान नहीं रहता कि उसके आस-पास कौन है। फिर शिकारी बहुत धीरे से अपना जाल फेकता है और कू-कू कोयल को पकड़ लेता है। जाल में फँसते ही कू-कू कोयल कहती है धत-तेरी कि ये क्या कर बैठी मैं। जादूगर ने कहा भी था मुझसे कि आपने आस-पास के लोगों से बच कर रहना। अब देखो मैंने अपनी लापरवाही के चक्कर में क्या कर लिया। जाल में फंस गई।
 
इधर कू-कू कोयल को ले जाते हुए शिकारी अपने मन में सोचता है कि क्यों न इस कोयल को राजा को ही दे दूँ। अगर राजा को दे दिया तो राजा इसके बदले मुझे राजमहल में रहने देंगे। फिर पूरी ज़िंदगी आराम से बीतेगी मेरी। हाँ यही सही रहेगा। ऐसा सोचते हुए शिकारी राजा के पास पहुंचता है और उसे सारी बात बताता है। राजा कोयल को लेने और उसके बदले शिकारी को अपने राज्य महल में रहने देने की बात मान जाता है। तभी वहाँ पर राजा का मंत्री आता है और उससे बोलता है कि महाराज, ये शिकारी झूठ बोल रहा। आपने कभी सुना या देखा भी है क्या किसी चिड़ियाँ के मुहँ से सोना निकले। भई मैंने तो नहीं देखा। मंत्री कि बात सुन कर राजा शिकारी से कहता है कि पहले मुझे सोना दिखाओ। आखिर ये कोयल कैसे सोना निकालती है। मैं भी तो देखूँ। राजा की बात सुन कर शिकारी कोयल से गाना गाने को कहता है। लेकिन कोयल तो इस बार कोई गलती नहीं करना चाहती थी, इसलिए वो अपना मुहँ ही नहीं खोलती। शिकारी के बहुत बार कहने पर भी जब कोयल गाना नहीं गाती तब राजा को ये सब देख कर गुस्सा आ जाता है और वो अपने सिपाहियों से कह कर कोयल को जाल से निकाल देता है और शिकारी को पकड़ लेता है। जाल से निकलते ही कोयल गाना गाना शुरू करती है कू-कू मैं हूँ कू-कू कोयल गाती हूँ मीठा-मीठा सुंदर गाना, जिसे सुनकर तुम सब मेरे पास और फिर तभी उसके मुहँ से फिर से पानी नीचे गिरता है जो सोने में बदल जाता है। राजा ये सब बस देखता ही रह जाता है और अपने सिपाहियों से कोयल को पकड़ने को कहता है। लेकिन इस बार कू-कू कोयल पहले जैसी कोई भी गलती नहीं करती है और वहाँ से उड़ती हुई सीधे ढोलकपुर जंगल पहुँचती है।    
\end{hindi}

\section{Nouns and verbs identification}
The first step is to identify the verbs and (common) nouns from the text. 
The table for that is given below.

\begin{table}[h]
\begin{tabular}{|l|l|l|l|l|l|l|}
\hline
\begin{hindi} बार \end{hindi} & \begin{hindi} जंगल \end{hindi} & \begin{hindi} कोयल \end{hindi} & \begin{hindi} मीठा \end{hindi} & \begin{hindi} गा \end{hindi} & \begin{hindi} हो \end{hindi} & \begin{hindi} सुन \end{hindi} \\ \hline 
\begin{hindi} आ \end{hindi} & \begin{hindi} जा \end{hindi} & \begin{hindi} आवाज \end{hindi} & \begin{hindi} सुना \end{hindi} & \begin{hindi} दा \end{hindi} & \begin{hindi} कर \end{hindi} & \begin{hindi} जादूगर \end{hindi} \\ \hline 
\begin{hindi} कह \end{hindi} & \begin{hindi} सो \end{hindi} & \begin{hindi} बोल \end{hindi} & \begin{hindi} समय \end{hindi} & \begin{hindi} पा \end{hindi} & \begin{hindi} गिर \end{hindi} & \begin{hindi} बूंद \end{hindi} \\ \hline 
\begin{hindi} बदल \end{hindi} & \begin{hindi} बात \end{hindi} & \begin{hindi} ख्याल \end{hindi} & \begin{hindi} रख \end{hindi} & \begin{hindi} लोग \end{hindi} & \begin{hindi} तरह \end{hindi} & \begin{hindi} नुकसान \end{hindi} \\ \hline 
\begin{hindi} चला \end{hindi} & \begin{hindi} लग \end{hindi} & \begin{hindi} अपना \end{hindi} & \begin{hindi} देख \end{hindi} & \begin{hindi} उठा \end{hindi} & \begin{hindi} लगा \end{hindi} & \begin{hindi} पहुँच \end{hindi} \\ \hline 
\begin{hindi} शिकारी \end{hindi} & \begin{hindi} पेड़ \end{hindi} & \begin{hindi} बैठ \end{hindi} & \begin{hindi} पहुंच \end{hindi} & \begin{hindi} ले \end{hindi} & \begin{hindi} मन \end{hindi} & \begin{hindi} सोच \end{hindi} \\ \hline 
\begin{hindi} आसमान \end{hindi} & \begin{hindi} बारिश \end{hindi} & \begin{hindi} नजर \end{hindi} & \begin{hindi} वाला \end{hindi} & \begin{hindi} निकल \end{hindi} & \begin{hindi} देरी \end{hindi} & \begin{hindi} पकड़ \end{hindi} \\ \hline 
\begin{hindi} बन \end{hindi} & \begin{hindi} खो \end{hindi} & \begin{hindi} जाल \end{hindi} & \begin{hindi} फेक \end{hindi} & \begin{hindi} फँस \end{hindi} & \begin{hindi} बच \end{hindi} & \begin{hindi} लापरवाही \end{hindi} \\ \hline 
\begin{hindi} चक्कर \end{hindi} & \begin{hindi} राजा \end{hindi} & \begin{hindi} दे \end{hindi} & \begin{hindi} बदला \end{hindi} & \begin{hindi} राजमहल \end{hindi} & \begin{hindi} आराम \end{hindi} & \begin{hindi} बीत \end{hindi} \\ \hline 
\begin{hindi} बता \end{hindi} & \begin{hindi} राज्य \end{hindi} & \begin{hindi} महल \end{hindi} & \begin{hindi} मान \end{hindi} & \begin{hindi} मंत्री \end{hindi} & \begin{hindi} महाराज \end{hindi} & \begin{hindi} झूठ \end{hindi} \\ \hline 
\begin{hindi} भई \end{hindi} & \begin{hindi} दिखा \end{hindi} & \begin{hindi} निकाल \end{hindi} & \begin{hindi} गल \end{hindi} & \begin{hindi} करना \end{hindi} & \begin{hindi} चाह \end{hindi} & \begin{hindi} खोल \end{hindi} \\ \hline 
\begin{hindi} गुस्सा \end{hindi} & \begin{hindi} सिपाही \end{hindi} & \begin{hindi} शुरू \end{hindi} & \begin{hindi} होना \end{hindi} & & & \\ \hline

\end{tabular}

\end{table}

\section{Semantic Features}
Assigning semantic features to these nouns and verbs.

\begin{enumerate}
\item \begin{hindi} बार \end{hindi} (instance) \\
Meaning: An occurrence or a specific time an event happens. \\
Semantic features: +temporal +countable

\item \begin{hindi} जंगल \end{hindi} (forest) \\
Meaning: A dense area covered with trees. \\
Semantic features: +natural +dense

\item \begin{hindi} कोयल \end{hindi} (cuckoo) \\
Meaning: A bird known for its melodious call. \\
Semantic features: +avian -human

\item \begin{hindi} मीठा \end{hindi} (sweet) \\
Meaning: A food item having a pleasant, sugary taste. \\
Semantic features: +taste +food

\item \begin{hindi} गा \end{hindi} (sing) \\
Meaning: To vocalize a tune or song. \\
Semantic features: +vocal +musical

\item \begin{hindi} हो \end{hindi} (be) \\
Meaning: To exist or occur. \\
Semantic features: +existence +state

\item \begin{hindi} सुन \end{hindi} (listen) \\
Meaning: To pay attention to sounds. \\
Semantic features: +auditory +perceptual

\item \begin{hindi} जा \end{hindi} (go) \\
Meaning: To move from one place to another. \\
Semantic features: +movement +directive

\item \begin{hindi} आवाज \end{hindi} (voice) \\
Meaning: The sound produced by a person or animal. \\
Semantic features: +auditory +expressive

\item \begin{hindi} सुना \end{hindi} (heard) \\
Meaning: Having listened or been made aware by sound. \\
Semantic features: +auditory +past

\item \begin{hindi} दा \end{hindi} (give) \\
Meaning: To bestow or hand over something. \\
Semantic features: +transference +command

\item \begin{hindi} कर \end{hindi} (do) \\
Meaning: To perform an action or execute a task. \\
Semantic features: +action +imperative

\item \begin{hindi} जादूगर \end{hindi} (magician) \\
Meaning: A person who practices magic or illusions. \\
Semantic features: +magical +performer

\item \begin{hindi} कह \end{hindi} (say) \\
Meaning: To utter words or express verbally. \\
Semantic features: +verbal +directive

\item \begin{hindi} सो \end{hindi} (sleep) \\
Meaning: To sleep. \\
Semantic features: +transitional +resultative

\item \begin{hindi} बोल \end{hindi} (speak) \\
Meaning: To articulate words; to express verbally. \\
Semantic features: +verbal +expressive

\item \begin{hindi} समय \end{hindi} (time) \\
Meaning: The ongoing sequence in which events occur. \\
Semantic features: +temporal +abstract

\item \begin{hindi} पा \end{hindi} (get) \\
Meaning: To acquire or obtain something. \\
Semantic features: +acquisition +action

\item \begin{hindi} गिर \end{hindi} (fall) \\
Meaning: To descend freely under the force of gravity. \\
Semantic features: +motion +gravity

\item \begin{hindi} बूंद \end{hindi} (drop) \\
Meaning: A tiny particle of liquid. \\
Semantic features: +liquid +minuscule

\item \begin{hindi} बदल \end{hindi} (change) \\
Meaning: To make or become different; to alter. \\
Semantic features: +transformation +action

\item \begin{hindi} बात \end{hindi} (matter) \\
Meaning: A subject, topic, or discussion point. \\
Semantic features: +communication +abstract

\item \begin{hindi} ख्याल \end{hindi} (thought) \\
Meaning: An idea, notion, or reflection. \\
Semantic features: +cognitive +abstract

\item \begin{hindi} रख \end{hindi} (keep) \\
Meaning: To maintain possession or retain. \\
Semantic features: +retention +action

\item \begin{hindi} लोग \end{hindi} (people) \\
Meaning: Human beings considered collectively. \\
Semantic features: +social +plural

\item \begin{hindi} तरह \end{hindi} (manner) \\
Meaning: A way or style of doing something; a kind. \\
Semantic features: +categorical +comparative

\item \begin{hindi} नुकसान \end{hindi} (damage) \\
Meaning: Harm or loss suffered by someone or something. \\
Semantic features: +negative +destructive

\item \begin{hindi} चला \end{hindi} (went) \\
Meaning: Moved or proceeded from one place to another (past tense). \\
Semantic features: +movement +past

\item \begin{hindi} लग \end{hindi} (attach/seem) \\
Meaning: To affix or appear in a certain way. \\
Semantic features: +attachment +perceptual

\item \begin{hindi} अपना \end{hindi} (own) \\
Meaning: Belonging to oneself; personal. \\
Semantic features: +possessive +reflexive

\item \begin{hindi} देख \end{hindi} (see) \\
Meaning: To perceive with the eyes. \\
Semantic features: +visual +perceptual

\item \begin{hindi} उठा \end{hindi} (lifted) \\
Meaning: Moved upward or raised something. \\
Semantic features: +elevation +action

\item \begin{hindi} लगा \end{hindi} (applied/felt) \\
Meaning: To have been affixed or to have appeared in a certain manner. \\
Semantic features: +causative +subjective

\item \begin{hindi} पहुँच \end{hindi} (reach) \\
Meaning: To arrive at a destination. \\
Semantic features: +movement +goal

\item \begin{hindi} शिकारी \end{hindi} (hunter) \\
Meaning: A person who hunts animals. \\
Semantic features: +predatory +active

\item \begin{hindi} पेड़ \end{hindi} (tree) \\
Meaning: A perennial plant with a trunk and branches. \\
Semantic features: +botanical +stationary

\item \begin{hindi} बैठ \end{hindi} (sit) \\
Meaning: To rest in a seated position. \\
Semantic features: +stationary +posture

\item \begin{hindi} पहुंच \end{hindi} (reach) \\
Meaning: To arrive at or get to a place. \\
Semantic features: +arrival +goal

\item \begin{hindi} ले \end{hindi} (take) \\
Meaning: To grasp or seize something. \\
Semantic features: +acquisition +action

\item \begin{hindi} मन \end{hindi} (mind) \\
Meaning: The faculty of thought and consciousness. \\
Semantic features: +cognitive +abstract

\item \begin{hindi} सोच \end{hindi} (think) \\
Meaning: To form ideas or consider. \\
Semantic features: +cognitive +process

\item \begin{hindi} आसमान \end{hindi} (sky) \\
Meaning: The expanse over the earth; the heavens. \\
Semantic features: +celestial +vast

\item \begin{hindi} बारिश \end{hindi} (rain) \\
Meaning: Water that falls from clouds as droplets. \\
Semantic features: +weather +liquid

\item \begin{hindi} नजर \end{hindi} (sight) \\
Meaning: The faculty or act of seeing. \\
Semantic features: +visual +perceptual

\item \begin{hindi} वाला \end{hindi} (related to) \\
Meaning: Denoting association or possession. \\
Semantic features: +associative +descriptive

\item \begin{hindi} निकल \end{hindi} (emerge) \\
Meaning: To come out or appear. \\
Semantic features: +movement +emergence

\item \begin{hindi} देरी \end{hindi} (delay) \\
Meaning: A lapse of time that causes lateness. \\
Semantic features: +temporal +lagging

\item \begin{hindi} पकड़ \end{hindi} (catch) \\
Meaning: To seize or hold firmly. \\
Semantic features: +grasp +action

\item \begin{hindi} बन \end{hindi} (become/make) \\
Meaning: To transform or create a state or object. \\
Semantic features: +transformation +creation

\item \begin{hindi} खो \end{hindi} (lose) \\
Meaning: To misplace or fail to retain. \\
Semantic features: +loss +action

\item \begin{hindi} जाल \end{hindi} (net) \\
Meaning: A mesh structure used for trapping or catching. \\
Semantic features: +entrapment +structure

\item \begin{hindi} फेक \end{hindi} (throw) \\
Meaning: To hurl or toss something away. \\
Semantic features: +action +discard

\item \begin{hindi} फँस \end{hindi} (get trapped) \\
Meaning: To become caught or ensnared. \\
Semantic features: +constraint +action

\item \begin{hindi} बच \end{hindi} (save/escape) \\
Meaning: To avoid harm or remain unscathed. \\
Semantic features: +preservation +rescue

\item \begin{hindi} लापरवाही \end{hindi} (negligence) \\
Meaning: A lack of proper care or attention. \\
Semantic features: +careless +omission

\item \begin{hindi} चक्कर \end{hindi} (dizziness) \\
Meaning: A state of spinning or disorientation. \\
Semantic features: +rotational +disorienting

\item \begin{hindi} राजा \end{hindi} (king) \\
Meaning: A male ruler or sovereign. \\
Semantic features: +royal +authoritative

\item \begin{hindi} दे \end{hindi} (give) \\
Meaning: To provide or hand over something. \\
Semantic features: +transfer +action

\item \begin{hindi} बदला \end{hindi} (revenge) \\
Meaning: The act of retaliating or seeking retribution. \\
Semantic features: +retributive +emotional

\item \begin{hindi} राजमहल \end{hindi} (palace) \\
Meaning: A grand residence of a king or noble. \\
Semantic features: +royal +architectural

\item \begin{hindi} आराम \end{hindi} (rest) \\
Meaning: A state of relaxation or ease. \\
Semantic features: +relaxation +comfort

\item \begin{hindi} बीत \end{hindi} (pass) \\
Meaning: To elapse or move by in time. \\
Semantic features: +temporal +transitional

\item \begin{hindi} बता \end{hindi} (tell) \\
Meaning: To communicate information or narrate. \\
Semantic features: +communicative +directive

\item \begin{hindi} राज्य \end{hindi} (state) \\
Meaning: A political territory or kingdom. \\
Semantic features: +political +territorial

\item \begin{hindi} महल \end{hindi} (palace) \\
Meaning: A large, stately residence. \\
Semantic features: +architectural +luxurious

\item \begin{hindi} मान \end{hindi} (respect) \\
Meaning: Regard or esteem for someone. \\
Semantic features: +respectful +abstract

\item \begin{hindi} मंत्री \end{hindi} (minister) \\
Meaning: A government official or head of a department. \\
Semantic features: +political +administrative

\item \begin{hindi} महाराज \end{hindi} (great king) \\
Meaning: A highly revered or eminent monarch. \\
Semantic features: +royal +august

\item \begin{hindi} झूठ \end{hindi} (lie) \\
Meaning: An untrue statement or falsehood. \\
Semantic features: +deceptive +false

\item \begin{hindi} भई \end{hindi} (brother) \\
Meaning: A term of address for a male sibling or close friend. \\
Semantic features: +familial +colloquial

\item \begin{hindi} दिखा \end{hindi} (show) \\
Meaning: To display or exhibit visually. \\
Semantic features: +visual +expressive

\item \begin{hindi} निकाल \end{hindi} (remove) \\
Meaning: To extract or take something out. \\
Semantic features: +extraction +action

\item \begin{hindi} गल \end{hindi} (mistake) \\
Meaning: An error or fault. \\
Semantic features: +incorrect +cognitive

\item \begin{hindi} करना \end{hindi} (do) \\
Meaning: To perform an action or execute an activity. \\
Semantic features: +action +performative

\item \begin{hindi} चाह \end{hindi} (desire) \\
Meaning: To want or long for something. \\
Semantic features: +emotional +intentional

\item \begin{hindi} खोल \end{hindi} (open) \\
Meaning: To make accessible or reveal. \\
Semantic features: +accessibility +action

\item \begin{hindi} गुस्सा \end{hindi} (anger) \\
Meaning: A strong feeling of displeasure or hostility. \\
Semantic features: +emotional +intense

\item \begin{hindi} सिपाही \end{hindi} (soldier) \\
Meaning: A person serving in an armed force. \\
Semantic features: +military +disciplined

\item \begin{hindi} शुरू \end{hindi} (start) \\
Meaning: To begin or commence an action. \\
Semantic features: +initiation +temporal

\item \begin{hindi} होना \end{hindi} (become/exist) \\
Meaning: To come into existence or occur. \\
Semantic features: +existence +process
\end{enumerate}

\section{WordNet Categorization}
Each of the words are put through the IndoWordNet website and their synonyms, hyponynms, hypernymys and meronymys are put below (if they exist in the database).

\begin{enumerate}

\item \begin{hindi}बार\end{hindi} \\
Synonyms: \begin{hindi}दफ़ा, दफा, मरतबा\end{hindi} \\
Hypernyms: \begin{hindi}समय, काल, वक्त\end{hindi}

\item \begin{hindi}बन\end{hindi} \\
Synonyms: \begin{hindi}वन, बन, अरण्य\end{hindi} \\
Hypernyms: \begin{hindi}पेड़-पौधे, पेड़ पौधे, वनस्पति समूह\end{hindi} \\
Hyponyms: \begin{hindi}सदाबहार वन, सदाबहार जंगल, हरित कानन\end{hindi} \\
Meronyms: \begin{hindi}वनस्पति, पेड़-पौधा\end{hindi}

\item \begin{hindi}कोयल\end{hindi} \\
Synonyms: \begin{hindi}कोकिला, कोकिल, पिक\end{hindi} \\
Hypernyms: \begin{hindi}गायक पक्षी\end{hindi}

\item \begin{hindi}मीठा\end{hindi} \\
Synonyms: \begin{hindi}मधुर, मिष्ट\end{hindi}

\item \begin{hindi}आ\end{hindi} \\
Synonyms: \begin{hindi}स्वराक्षर आ, स्वर अक्षर आ\end{hindi} \\
Hypernyms: \begin{hindi}स्वर, स्वर अक्षर, स्वराक्षर\end{hindi}

\item \begin{hindi}जा\end{hindi} \\
Synonyms: \begin{hindi}जा, देरानी, दिरानी\end{hindi} \\
Hypernyms: \begin{hindi}रिश्तेदार, नातेदार, संबंधी\end{hindi}

\item \begin{hindi}आवाज\end{hindi} \\
Synonyms: \begin{hindi}आवाज, बोली, कंठ स्वर\end{hindi} \\
Hypernyms: \begin{hindi}ध्वनि, आवाज़, आवाज\end{hindi} \\
Hyponyms: \begin{hindi}फुसफुसाहट, फुसफुस, फुस-फुस\end{hindi}

\item \begin{hindi}सुना\end{hindi} \\
Synonyms: \begin{hindi}सुना, श्रुत, आश्रुत\end{hindi}

\item \begin{hindi}कर\end{hindi} \\
Synonyms: \begin{hindi}टैक्स, महसूल, शालिक\end{hindi} \\
Hypernyms: \begin{hindi}धन-दौलत, दौलत, धन\end{hindi} \\
Hyponyms: \begin{hindi}लगान, मालगुजारी, मालगुज़ारी\end{hindi}

\item \begin{hindi}जादूगर\end{hindi} \\
Synonyms: \begin{hindi}बाज़ीगर, बाजीगर, ऐंद्रजालिक\end{hindi} \\
Hypernyms: \begin{hindi}व्यक्ति, मनुष्य, मानस\end{hindi} \\
Hyponyms: \begin{hindi}जादूगरनी, बाजीगरनी\end{hindi}

\item \begin{hindi}सो\end{hindi} \\
Synonyms: \begin{hindi}साल, सोल, सोह\end{hindi} \\
Hypernyms: \begin{hindi}संगीत स्वर, स्वर, सुर\end{hindi}

\item \begin{hindi}बोल\end{hindi} \\
Synonyms: \begin{hindi}प्रबंध, प्रबन्ध\end{hindi} \\
Hypernyms: \begin{hindi}ध्वनि, आवाज़, आवाज\end{hindi} \\
Hyponyms: \begin{hindi}यति\end{hindi}

\item \begin{hindi}समय\end{hindi} \\
Synonyms: \begin{hindi}काल, वक्त, वक़्त\end{hindi} \\
Hypernyms: \begin{hindi}बोध, संज्ञान, ज्ञान\end{hindi} \\
Hyponyms: \begin{hindi}अंतराल, अन्तराल, गोशा\end{hindi}

\item \begin{hindi}बूंद\end{hindi} \\
Synonyms: \begin{hindi}कण, क़तरा, कतरा\end{hindi} \\
Hypernyms: \begin{hindi}भाग, हिस्सा, टुकड़ा\end{hindi} \\
Hyponyms: \begin{hindi}छींटा, छींट\end{hindi}

\item \begin{hindi}पा\end{hindi} \\
Synonyms: \begin{hindi}पंचम स्वर, पा, प\end{hindi} \\
Hypernyms: \begin{hindi}संगीत स्वर, स्वर, सुर\end{hindi}

\item \begin{hindi}गिर\end{hindi} \\
Synonyms: \begin{hindi}पहाड़, गिरि, शैल\end{hindi} \\
Hypernyms: \begin{hindi}प्राकृतिक वस्तु, नैसर्गिक वस्तु\end{hindi} \\
Hyponyms: \begin{hindi}हिमालय, गिरिपति, गिरीश\end{hindi} \\
Meronyms: \begin{hindi}शिखर, चोटी, शिखा\end{hindi}

\item \begin{hindi}बात\end{hindi} \\
Synonyms: \begin{hindi}ज़िक्र, उल्लेख, चर्चा\end{hindi} \\
Hypernyms: \begin{hindi}काम, कार्य, कर्म\end{hindi} \\
Hyponyms: \begin{hindi}आभोग\end{hindi}

\item \begin{hindi}ख्याल\end{hindi} \\
Synonyms: \begin{hindi}ध्यान, ख़याल, खयाल\end{hindi} \\
Hypernyms: \begin{hindi}ज्ञान, जानकारी, प्रतीति\end{hindi} \\
Hyponyms: \begin{hindi}चटखारा, चटकारा\end{hindi}

\item \begin{hindi}लोग\end{hindi} \\
Synonyms: \begin{hindi}जन, लोक, जनमानस\end{hindi} \\
Hypernyms: \begin{hindi}समुदाय, समूह, झुंड\end{hindi} \\
Hyponyms: \begin{hindi}परजन, गैर, ग़ैर\end{hindi} \\
Meronyms: \begin{hindi}व्यक्ति, मनुष्य, मानस\end{hindi}

\item \begin{hindi}तरह\end{hindi} \\
Synonyms: \begin{hindi}तरह, किस्म, क़िस्म\end{hindi} \\
Hypernyms: \begin{hindi}वर्ग, श्रेणी, तबका\end{hindi} \\
Hyponyms: \begin{hindi}आंगिक, आँगिक, आंगिक अभिनय\end{hindi}

\item \begin{hindi}नुकसान\end{hindi} \\
Synonyms: \begin{hindi}नुक़सान, नुकसान, हानि\end{hindi} \\
Hypernyms: \begin{hindi}नुकसान, नुक़सान, हानि\end{hindi}

\item \begin{hindi}अपना\end{hindi} \\
Synonyms: \begin{hindi}निजी, ज़ाती, अपना\end{hindi}

\item \begin{hindi}उठा\end{hindi} \\
Synonyms: \begin{hindi}उठा, उठा हुआ, खड़ा\end{hindi}

\item \begin{hindi}लगा\end{hindi} \\
Synonyms: \begin{hindi}लगा हुआ\end{hindi}

\item \begin{hindi}पहुंच\end{hindi} \\
Synonyms: \begin{hindi}दौड़, पहुंच\end{hindi} \\
Hypernyms: \begin{hindi}सीमा, हद, मर्यादा\end{hindi}

\item \begin{hindi}मन\end{hindi} \\
Synonyms: \begin{hindi}चित्त, चित, मानस\end{hindi} \\
Hypernyms: \begin{hindi}अमूर्त वस्तु, आकारहीन वस्तु, निरायाम वस्तु\end{hindi}

\item \begin{hindi}सोच\end{hindi} \\
Synonyms: \begin{hindi}नजरिया, नज़रिया, सोच\end{hindi} \\
Hypernyms: \begin{hindi}बोध, संज्ञान, ज्ञान\end{hindi} \\
Hyponyms: \begin{hindi}पहचान, नज़र, परख\end{hindi}

\item \begin{hindi}आसमान\end{hindi} \\
Synonyms: \begin{hindi}आसमान, गगन, नभ\end{hindi} \\
Hypernyms: \begin{hindi}प्राकृतिक वस्तु, नैसर्गिक वस्तु\end{hindi} \\
Hyponyms: \begin{hindi}अंतरिक्ष, अन्तरिक्ष, अंतरीक\end{hindi} \\
Meronyms: \begin{hindi}बादल, मेघ, मेघा\end{hindi}

\item \begin{hindi}बारिश\end{hindi} \\
Synonyms: \begin{hindi}वर्षा, बरखा, पावस\end{hindi} \\
Hypernyms: \begin{hindi}जल, पानी, नीर\end{hindi}

\item \begin{hindi}निकल\end{hindi} \\
Synonyms: \begin{hindi}निकल\end{hindi} \\
Hypernyms: \begin{hindi}रासायनिक तत्व, रासायनिक तत्त्व, रसायनिक तत्व\end{hindi}

\item \begin{hindi}देरी\end{hindi} \\
Synonyms: \begin{hindi}देरी, विलंब, विलम्ब\end{hindi} \\
Hypernyms: \begin{hindi}समय, काल, वक्त\end{hindi} \\
Hyponyms: \begin{hindi}लालफ़ीता, लाल फ़ीता, लालफीता\end{hindi}

\item \begin{hindi}खो\end{hindi} \\
Synonyms: \begin{hindi}\end{hindi} \\
Hypernyms: \begin{hindi}ध्वनि, आवाज़, आवाज\end{hindi}

\item \begin{hindi}जाल\end{hindi} \\
Synonyms: \begin{hindi}पाश, आनाय\end{hindi} \\
Hypernyms: \begin{hindi}औजार, औज़ार, उपकरण\end{hindi} \\
Hyponyms: \begin{hindi}महाजाल, जंजाल, बड़ा-जाल\end{hindi}

\item \begin{hindi}फेक\end{hindi} \\
Synonyms: \begin{hindi}फेक शहर\end{hindi} \\
Hypernyms: \begin{hindi}शहर, नगर, नगरी\end{hindi}

\item \begin{hindi}बच\end{hindi} \\
Synonyms: \begin{hindi}बच, उग्रगंधा, उग्रगन्धा\end{hindi} \\
Hypernyms: \begin{hindi}पौधा, पौदा\end{hindi}

\item \begin{hindi}लापरवाही\end{hindi} \\
Synonyms: \begin{hindi}लापरवाही, बेपरवाही, असावधानता\end{hindi} \\
Hypernyms: \begin{hindi}मानसिक अवस्था, मनोदशा, मनःस्थिति\end{hindi}

\item \begin{hindi}चक्कर\end{hindi} \\
Synonyms: \begin{hindi}\end{hindi} \\
Hypernyms: \begin{hindi}अवस्था, दशा, स्थिति\end{hindi}

\item \begin{hindi}राजा\end{hindi} \\
Synonyms: \begin{hindi}अवनीश, नराधिप, नरेश\end{hindi} \\
Hypernyms: \begin{hindi}शासक, नियंता, नियन्ता\end{hindi} \\
Hyponyms: \begin{hindi}सम्राट, शहंशाह, शाहंशाह\end{hindi}

\item \begin{hindi}बदला\end{hindi} \\
Synonyms: \begin{hindi}बदला, इंतकाम, इंतक़ाम\end{hindi} \\
Hypernyms: \begin{hindi}काम, कार्य, कर्म\end{hindi}

\item \begin{hindi}महल\end{hindi} \\
Synonyms: \begin{hindi}महल, प्रासाद, राजप्रासाद\end{hindi} \\
Hypernyms: \begin{hindi}भवन, इमारत, वास्तु\end{hindi} \\
Hyponyms: \begin{hindi}ताजमहल, ताज\end{hindi}

\item \begin{hindi}आराम\end{hindi} \\
Synonyms: \begin{hindi}आराम, अराम, बिसराम\end{hindi} \\
Hypernyms: \begin{hindi}काम, कार्य, कर्म\end{hindi} \\
Hyponyms: \begin{hindi}सोना, शयन, सयन\end{hindi}

\item \begin{hindi}राज्य\end{hindi} \\
Synonyms: \begin{hindi}राज्य, प्रांत, प्रान्त\end{hindi} \\
Hypernyms: \begin{hindi}क्षेत्र, इलाका, इलाक़ा\end{hindi} \\
Hyponyms: \begin{hindi}असम, आसाम\end{hindi} \\
Meronyms: \begin{hindi}मंडल, संभाग, मण्डल\end{hindi}

\item \begin{hindi}मान\end{hindi} \\
Synonyms: \begin{hindi}आदर, इज़्ज़त, इज्जत\end{hindi} \\
Hypernyms: \begin{hindi}सद्व्यवहार, सदाचार, सद् व्यवहार\end{hindi} \\
Hyponyms: \begin{hindi}स्वागत, अगवानी, अगवाई\end{hindi}

\item \begin{hindi}मंत्री\end{hindi} \\
Synonyms: \begin{hindi}मन्त्री, मिनिस्टर, वजीर\end{hindi} \\
Hypernyms: \begin{hindi}अधिकारी\end{hindi} \\
Hyponyms: \begin{hindi}गृहमंत्री, गृह-मंत्री, गृह मंत्री\end{hindi}

\item \begin{hindi}महाराज\end{hindi} \\
Synonyms: \begin{hindi}महाराजा, महाराज, चक्रवर्ती\end{hindi} \\
Hypernyms: \begin{hindi}राजा, अवनीश, नराधिप\end{hindi}

\item \begin{hindi}झूठ\end{hindi} \\
Synonyms: \begin{hindi}असत्य, गलत, ग़लत\end{hindi}

\item \begin{hindi}भई\end{hindi} \\
Hypernyms: \begin{hindi}शब्द, लफ़्ज़, लफ्ज\end{hindi}

\item \begin{hindi}करना\end{hindi} \\
Synonyms: \begin{hindi}काम करना\end{hindi} \\
Hypernyms: \begin{hindi}काम, कार्य, कर्म\end{hindi}

\item \begin{hindi}चाह\end{hindi} \\
Synonyms: \begin{hindi}अभिलाषा, आकांक्षा, ख्वाहिश\end{hindi} \\
Hypernyms: \begin{hindi}मनोभाव, मानसिक भाव, मनोभावना\end{hindi} \\
Hyponyms: \begin{hindi}जिज्ञासा, उत्कंठा, उत्कण्ठा\end{hindi}

\item \begin{hindi}खोल\end{hindi} \\
Synonyms: \begin{hindi}अपाहिज, अंगहीन, अपंग\end{hindi}

\item \begin{hindi}गुस्सा\end{hindi} \\
Synonyms: \begin{hindi}गुस्सा, आक्रोश, कोप\end{hindi} \\
Hypernyms: \begin{hindi}मनोभाव, मानसिक भाव, मनोभावना\end{hindi} \\
Hyponyms: \begin{hindi}खीज, झुँझलाहट, कुढ़न\end{hindi}

\item \begin{hindi}सिपाही\end{hindi} \\
Synonyms: \begin{hindi}योद्धा, योधा, जोधा\end{hindi} \\
Hypernyms: \begin{hindi}व्यक्ति, मनुष्य, मानस\end{hindi} \\
Hyponyms: \begin{hindi}सीमा रक्षक, सीमापाल, सीमारक्षक\end{hindi}

\item \begin{hindi}शुरू\end{hindi} \\
Synonyms: \begin{hindi}आरम्भ, शुरुआत, शुरुवात\end{hindi} \\
Hypernyms: \begin{hindi}क्रिया\end{hindi} \\
Hyponyms: \begin{hindi}उद्घाटन\end{hindi}

\item \begin{hindi}होना\end{hindi} \\
Hyponyms: \begin{hindi}लूटना\end{hindi} 

\end{enumerate}

\section{Comparison with English WordNet}
10 nouns and verbs each are chosen to demonstrate the differences between the wordnets of both the languages. The translation of the chosen words is put through and the result is displayed below.

\begin{enumerate}

\item sing \\
Synonyms:  vocalize, chirp

\item show \\
Synonyms:  show \\
Hypernyms: act,  human action,  human activity \\
Hyponyms: tinsel

\item listen \\
Synonyms:  hear,  take heed

\item go \\
Synonyms:  fling,  go,  pass \\
Hypernyms: attempt,  effort,  endeavor

\item do \\
Synonyms:  do,  brawl \\
Hypernyms: banquet,  feast

\item say \\
Synonyms:  tell,  enjoin,  say

\item speak \\
Synonyms:  speak,  utter,  mouth

\item fall \\
Synonyms:  fall,  surrender \\
Hypernyms: 

\item see \\
Hypernyms: site,  land site

\item start \\
Synonyms:  start,  commencement \\
Hypernyms: origin,  origination,  inception

\item forest \\
Synonyms:  wood,  woods \\
Hypernyms: group,  grouping \\
Meronyms: plant,  flora,  plant life

\item cuckoo \\
Hypernyms: birds, avians

\item voice \\
Synonyms:  vocalization,  vocalisation,  vocalism \\
Hypernyms: sound \\
Hyponyms: whisper,  whispering,  susurration

\item magician \\
Synonyms:  prestidigitator,  conjurer,  conjuror \\
Hypernyms: person,  individual,  someone \\
Hyponyms: enchantress,  witch

\item time \\
Hypernyms: continuance,  continuation

\item drop \\
Synonyms:  drop,  drop-off \\
Hypernyms: rock,  stone

\item talk \\
Synonyms:  talk of the town \\
Hypernyms: happening,  occurrence,  occurrent

\item people \\
Synonyms:  people \\
Hypernyms: inhabitant,  habitant,  dweller \\
Hyponyms: national,  subject

\item sky \\
Hypernyms: natural object \\
Hyponyms: outer space \\
Meronyms: cloud

\item king \\
Hypernyms: chessman,  chess piece 
\end{enumerate}

Both networks use hierarchies. For instance, the Hindi entry for \begin{hindi} बार \end{hindi} lists hypernyms like \begin{hindi} समय, काल, वक्त \end{hindi} (all referring to time), much like the English WordNet groups “time” under broader concepts like “continuance” or “duration.” This shows a shared understanding that even if cultural nuances differ, there is an underlying need to classify words within a broader conceptual framework.

Synonyms: In the Hindi entry for \begin{hindi} बन \end{hindi}(\begin{hindi} “बन” \end{hindi} meaning forest), synonyms such as \begin{hindi}वन, अरण्य \end{hindi}capture alternate expressions of the same concept. Similarly, English WordNet groups “forest” with related words like “woods.”
Hypernyms and Hyponyms: For example, \begin{hindi}बन \end{hindi}includes hypernyms like \begin{hindi}पेड़-पौधे, वनस्पति समूह\end{hindi} and hyponyms such as \begin{hindi}सदाबहार वन, हरित कानन\end{hindi}. This mirrors the English method, where “forest” might have hypernyms related to natural regions and hyponyms specifying subtypes.
Meronyms: The Hindi entry for \begin{hindi}बन \end{hindi}lists meronyms like\begin{hindi} वनस्पति, पेड़-पौधा, \end{hindi}indicating the living parts of a forest. Similarly, English WordNet may list “flora” or “plant life” as parts of a forest.

\begin{hindi}कोयल\end{hindi} (Cuckoo):
The Hindi WordNet entry for\begin{hindi} कोयल\end{hindi} provides synonyms such as \begin{hindi}कोकिला, पिक \end{hindi}and a hypernym \begin{hindi}गायक पक्षी \end{hindi}(singing bird). This entry illustrates a culturally embedded perspective that emphasizes the bird’s melodious nature—a feature that might be only implicitly captured in the English “cuckoo”.

\begin{hindi}जादूगर\end{hindi} (Magician):
The Hindi entry for \begin{hindi}जादूगर\end{hindi} lists culturally resonant synonyms like\begin{hindi} बाज़ीगर \end{hindi}and \begin{hindi}ऐंद्रजालिक\end{hindi} and distinguishes further by including hyponyms (e.g., \begin{hindi}जादूगरनी \end{hindi}for a female magician). In English, while the entry for “magician” might list synonyms such as “prestidigitator” or “conjuror,” the explicit gender differentiation seen in the Hindi entry reflects different cultural nuances in lexical usage.

\begin{hindi}समय\end{hindi} (Time):
The Hindi entry for समय provides synonyms such as \begin{hindi}काल, वक्त, वक़्त\end{hindi} and hypernyms like \begin{hindi}बोध, संज्ञान, ज्ञान,\end{hindi} which not only capture the temporal aspect but also tie in cognitive elements. The English “time” often gets abstractly categorized under “continuance” or “duration” without as strong an association with cognitive terms.

\begin{hindi}बूंद \end{hindi}(Drop):
In Hindi, \begin{hindi}बूंद\end{hindi} is connected with synonyms like \begin{hindi}कण, क़तरा, कतरा\end{hindi} and hypernyms that suggest it is a part or piece (e.g., \begin{hindi}भाग, हिस्सा, टुकड़ा\end{hindi}). This detailed breakdown into related parts shows a fine-grained lexicalization that might be more generalized in English, where “drop” may simply be classified as a small liquid particle.
\\

Example: "\begin{hindi}बन\end{hindi}" (Forest) vs. "forest"\\
Hindi Entry for \begin{hindi}बन\end{hindi}:\\
Synonyms: \begin{hindi}वन, अरण्य\end{hindi}\\
Hypernyms: \begin{hindi}पेड़-पौधे, वनस्पति समूह\end{hindi}\\
Hyponyms: \begin{hindi}सदाबहार वन, हरित कानन\end{hindi}\\
Meronyms: \begin{hindi}वनस्पति, पेड़-पौधा\end{hindi}\\
English Entry for forest:\\
Synonyms: wood, woods\\
Hypernyms: natural area, wooded area\\
Meronyms: flora, plant life\\
Here, both entries describe the general concept of a forest along with its parts, yet the Hindi entry goes further by including specific subtypes of forests and explicitly listing the constituent plant life.\\

Example: "\begin{hindi}समय\end{hindi}" (Time) vs. "time"\\
Hindi Entry for \begin{hindi}समय\end{hindi}:\\
Synonyms: \begin{hindi}काल, वक्त, वक़्त\end{hindi}\\
Hypernyms: \begin{hindi}बोध, संज्ञान, ज्ञान\end{hindi}\\
Hyponyms: \begin{hindi}अंतराल, अन्तराल, गोशा\end{hindi}\\
English Entry for time:\\
Synonyms: duration, period, interval\\
Hypernyms: continuance, continuation\\
Hyponyms: moment, era, period\\
This comparison illustrates that while both languages abstract the idea of time, the Hindi entry integrates cognitive or perceptual associations (\begin{hindi}बोध, संज्ञान)\end{hindi} that are less pronounced in the English classification.\\

Example: \begin{hindi}"जादूगर"\end{hindi} (Magician) vs. "magician"\\
Hindi Entry for \begin{hindi}जादूगर\end{hindi}:\\
Synonyms: \begin{hindi}बाज़ीगर, ऐंद्रजालिक\end{hindi}\\
Hypernyms: \begin{hindi}व्यक्ति, मनुष्य, मानस\end{hindi}\\
Hyponyms: \begin{hindi}जादूगरनी, बाजीगरनी\end{hindi}\\
English Entry for magician:\\
Synonyms: prestidigitator, conjurer, conjuror\\
Hypernyms: person, individual, someone\\
Hyponyms: enchantress, witch\\
In this example, both entries classify “magician” as a type of person. However, the Hindi entry explicitly provides forms for female magicians, reflecting a nuanced treatment of gender that may be handled differently in English.

\end{document}
