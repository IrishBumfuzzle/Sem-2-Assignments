\documentclass{article}
\usepackage{graphicx} % Required for inserting images
\usepackage{polyglossia}
\setmainlanguage{english}
\setotherlanguages{hindi}
\newfontfamily\devanagarifont[Script=Devanagari]{Noto Serif Devanagari}


\title{ITL Assignment 2\\Lost in Translation}
\author{Ojas Kataria}
\date{}

\begin{document}

\maketitle

\section{Task 1}

\subsection{The Jokes}
\begin{enumerate}
    \item Why don’t scientists trust atoms? Because they make up everything!
    \item I used to be a baker because I kneaded dough.
    \item The bicycle couldn't stand on its own because it was two-tired.
    \item What do you call a bear with no teeth? A gummy bear!
    \item Why did the scarecrow win an award? Because he was outstanding in his field!
\end{enumerate}
\subsection{Explanation}
\begin{enumerate}
    \item The joke plays on the polysemy of the phrase `make up everything' which has 2 meanings. One is that they are the building blocks of everything and the other is the act of constructing a lie/false statement. From the context of real-life atoms, we know that they are building blocks of everything, and the context of `not trusting' implies the lying meaning.
    \medskip\\
    \textbf{Translation}: \begin{hindi}{वैज्ञानिक परमाणुओं पर भरोसा क्यों नहीं करते? क्योंकि वे सब कुछ बनाते हैं!}\end{hindi} \\
    The meaning can be said to be \textit{partially} preserved since the word \begin{hindi}बनाते\end{hindi} in Hindi can also used for both the meanings but the usage of the word differs from English and hence the joke doesn't make sense anymore.

    \item The joke arises out of the homophony of the words kneaded and needed and the homonymy of the word dough meaning either the paste made from flour or money.
    Since a baker kneads the dough (to make bread) and also needs money, the double meaning of the sentence makes it a joke.
    One must know the context of how to make bread and that a person needs money in the real word to understand the two meanings. 
    \medskip\\
    \textbf{Translation}: \begin{hindi}
        मैं बेकर था क्योंकि मुझे आटा गूँधने की ज़रूरत थी।
    \end{hindi}\\
    The meaning is completely lost because both the homonymy and the homophony of the words dough and knead are not present in Hindi.

    \item The joke is based on the two homophones two, too, and the polysemy of the word tired, which means being lethargic or having two tires in the context of a vehicle.\medskip\\
    \textbf{Translation}: \begin{hindi}साइकिल खुद खड़ी नहीं हो सकी क्योंकि उसके दो टायर थे।\end{hindi}
    The joke completely lost its meaning since there is no homophony, and no polysemy of \begin{hindi}टायर\end{hindi}

    \item The organs teeth are attached to in animals are called \textit{gums}, hence the adjective \textit{gummy} for a toothless animal and a gummy bear is a small piece of candy. The phrase `gummy bear' is polysemous.
    \medskip\\
    \textbf{Translation}: \begin{hindi}दाँतों के बिना भालू को क्या कहते हैं? गमी बेयर!\end{hindi} \\
    Since there is no direct translation of the word \textit{gummy}, `gummy bear' can either be directly transliterated like above or translated into \begin{hindi}बिना दाँतों वाला भालू\end{hindi}. Regardless, both the senses of the phrase cannot be present at the same time, hence the losing of the humor (like me fr fr).

    \item A scarecrow has usage in a field for as the name suggests, \textit{scaring away crows} and outstanding means being extraordinary/special. The words `out standing' and `outstanding' are homophones. Also the word field is polysemous, it means an open area which can be used in a physical context `field of grass' or academic contexts such as `field of science'.
    \medskip\\
    \textbf{Translation}: \begin{hindi}
        बिजूका को इनाम क्यों मिला? क्योंकि वह अपने खेत में बाहर खड़ा था!
    \end{hindi}\\
    The joke just got converted into a literal statement with no ambiguity whatsoever. The second meaning of being great in his ``field" is lost.
\end{enumerate}

\newpage
\section{Task 2}
\textbf{Karma}\\
The word Karma is a Sanskrit term that literally translates to 'action', but the word has significant connotations of a spiritual and philosophical nature.

In its original context, the word refers to the phenomenon whereby the choices you make, have or make an effect on your future. How your choices can completely change your path, be it positive or negative. This principle is a pillar of Indian religious (especially Hinduism) and philosophical discussions, thought processes, and books. It is closely related to, what some religions believe in, the cycle of rebirth. That the choices you make in one life have an effect on your other lives.

The literal translation, `action' loses the meaning of the \textit{effect} your `actions' have in your life. It conveys only the physical deed but not the philosophical or ethical one nor the lasting effect the action actually has. It does not have the layers of meaning related to intention, ethical responsibility, and the long-term repercussions that are central to Karma. In Indian philosophical thought, the emphasis is not only on the deed itself but also on the underlying motivation and the ripple effects that extend beyond the immediate moment. This aspect of Karma is closely linked to the idea of `Dharma' (duty/righteousness), where actions are measured against the standards of what is defined as moral or ethical.

The original word has significant emotional nuance, a sense of cosmic justice, moral responsibility. The idea of Karma challenges people to reflect on the accumulative impact of their actions, emphasising that each choice contributes to their destiny.

This makes the term into a very intrinsically Indian term which when translated into English cannot encompass what it means to convey. Ironically, the word is so hard to translate that the Oxford English dictionary has included the term itself into its dictionary.

\end{document}
