\documentclass{article}
\usepackage{enumitem}
\usepackage{fontspec}
\setmainfont{BABEL Unicode} % Ensure Devanagari script support (use XeLaTeX/LuaLaTeX)

\title{ITL-2 Assignment 1 \\Types of Meaning}
\author{Ojas Kataria}
\date{}

\begin{document}
	\maketitle
	
	\begin{enumerate}[leftmargin=*]
		\item \textbf{Conceptual Meaning} 
		\begin{itemize}
			\item \textit{Definition}: Literal/dictionary meaning
			\item \textit{Hindi Example}: "कुत्ता" (kutta) refers literally to the animal "dog."
		\end{itemize}
		
		\item \textbf{Connotative Meaning}
		\begin{itemize}
			\item \textit{Definition}: Cultural/emotional associations
			\item \textit{Hindi Example}: Calling someone "कुत्ता" (kutta) implies derogatory traits like meanness or servility, drawing on negative cultural associations.
		\end{itemize}
		
		\item \textbf{Social Meaning}
		\begin{itemize}
			\item \textit{Definition}: Social context/formality markers
			\item \textit{Hindi Example}: Using "तुम" (tum, informal "you") vs. "आप" (aap, formal "you") reflects the speaker's relationship with the listener.
		\end{itemize}
		
		\item \textbf{Affective Meaning}
		\begin{itemize}
			\item \textit{Definition}: Emotional tone/attitude
			\item \textit{Hindi Example}:  "चुप रहो!" (chup raho! – "Shut up!") expresses anger, while "कृपया शांत रहें" (kripya shaant rahen – "Please stay quiet") is polite.
		\end{itemize}
		
		\item \textbf{Reflected Meaning}
		\begin{itemize}
			\item \textit{Definition}: Interconnected word senses
			\item \textit{Hindi Example}: "शेर" (sher) means "lion" but also connotes bravery, reflecting the lion's symbolic association with courage.
		\end{itemize}
		
		\item \textbf{Collocative Meaning}
		\begin{itemize}
			\item \textit{Definition}: Word-pairing associations
			\item \textit{Hindi Example}: "चाय की प्याली" (chai ki pyali – "cup of tea") is a natural collocation, unlike "चाय का गिलास" (chai ka glass), which is less typical.
		\end{itemize}
		
		\item \textbf{Thematic Meaning}
		\begin{itemize}
			\item \textit{Definition}: Focus through word order
			\item \textit{Hindi Example}:  "राम ने सीता को उपहार दिया" (Ram gave Sita a gift) emphasizes the subject, while "सीता को राम ने उपहार दिया" shifts focus to the recipient (Sita).
		\end{itemize}
	\end{enumerate}
	
\end{document}